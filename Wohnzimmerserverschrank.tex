\documentclass[10pt,a4paper]{article}%IEEEtran}
\usepackage[utf8]{inputenc}
\usepackage[ngerman]{babel}
\usepackage{amsmath}
\usepackage{amsfonts}
\usepackage{textcomp}
\usepackage{amssymb}
\usepackage{gensymb}
\usepackage{paralist}
\def\inch#1{#1''}
\def\ft#1{#1'\thinspace}

\date{\today}
\author{Matthias N.}
\title{\underline{W}ohn\underline{z}immer\underline{s}erver\underline{s}chrank (WZSS)}

\begin{document}


\maketitle

\section{Was stelle ich mir unter einem WZSS vor?}
Ein WZSS soll folgenden Problemen Abhilfe schaffen:
\begin{itemize}
\item In der Wohnumgebung stehen Rechner, die laut sind und deren Lüfter schnell verstopfen. 
\item Laserdrucker, die Toner in die Luft blasen.
\item Geräte die Brummen oder Lüfter besitzen
\item Geräte die Strom brauchen und nicht gut aussehen.
 \end{itemize}
Die Idee sieht einen Kasten der schallgedämpft, passiv oder zumindest lautlos gekühlt und abgeschlossen ist.


\section{Anforderungen an den WZSS}

\begin{itemize}
		\item Erzeugt (fast) keine Geräusche
		\item stark schalldämpfend
		\item 1\inch 9   Einbaumöglichkeit
		\item Tischhöhe (ca. 75 cm)
		\item leicht zugänglich (zB. für Drucker)
		\item durchdachte Kabelführung
		\item min 2000 Watt Wärmeableitung bei 15\degree C Wärmeunterschied
		\item modularisierbar (mehreren WZSS zusammensetzbar)
		\item Auslesen der Parameter und Ausgabe von Warnungen (Wärme innen, außen, Energieverbrauch)
\end{itemize}
\section{Welchen Umfang wir das WZSS einnehmen?}
Durch seine physikalische Größe und Materialaufwand ist dieses ein relative teures und sehr Mechanik lastiges Projekt.\\
Ich rechne mit ca. 350\texteuro  { }Kosten pro Schrank.\\ Für diese Projekt wäre es gut, wenn sich eine zwei bis drei Personen finden würden.  


\thispagestyle{empty}
\end{document}
