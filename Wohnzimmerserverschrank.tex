\documentclass[10pt,a4paper]{scrartcl}%IEEEtran}
\usepackage[utf8]{inputenc}
\usepackage[ngerman]{babel}
\usepackage{amsmath}
\usepackage{amsfonts}
\usepackage{textcomp}
\usepackage{amssymb}
\usepackage{gensymb}
\usepackage{paralist}
\def\inch#1{#1''}
\def\ft#1{#1'\thinspace}

\date{\today}
\author{Mattan}
\title{\underline{W}ohn\underline{z}immer\underline{s}erver\underline{s}chrank (WZSS)}

\begin{document}


\maketitle

\section{Was stelle ich mir unter einem WZSS vor?}
Ein WZSS soll folgenden Problemen Abhilfe schaffen:
Die Elektronik in der Wohnumgebung darf keine Emission in den Raum strahlen. 
\begin{itemize}
\item Kein Licht
\item Kein Schall
\item Bedienung möglich
\end{itemize}
Die Idee sieht einen Kasten der schallgedämpft, passiv oder zumindest lautlos gekühlt und abgeschlossen ist.


\section{Anforderungen an den WZSS}

\begin{itemize}
		\item Erzeugt keine Geräusche
		\item stark schalldämpfend
		\item \inch{5,25} Laufwerkeinbaumöglichkeit
		\item CD-Schachtgröße 
		\item maximal Tischhöhe (ca. 40 cm)
		\item leicht zugänglich (modular)
		\item durchdachte Kabelführung
		\item Wärmeableitung bei 15\degree C Wärmeunterschied
		\item modularisierbar (mehreren WZSS zusammensetzbar)
		\item Auslesen der Parameter und Ausgabe von Warnungen (Wärme innen, außen, Energieverbrauch)
\end{itemize}
\section{Welchen Umfang wir das WZSS einnehmen?}

Ich rechne mit circa 50 \texteuro Kosten pro Schrank. 
\section{Planung}
\subsection{Technische Daten}
\begin{itemize}
%\item \inch{5.25} bei 1 HE entspricht \textbf{Breite:}\ 482,6 mm \textbf{Höhe:}\ 44,45 mm \textbf{Tiefe:}\ 450 mm (innen)
\item CD-Laufwerk \textbf{Breit: } 146 mm \textbf{Höhe:\ } 41 mm \textbf{Tiefe:\ }165 mm - 200 mm 
\item HDF-Platte 300 mm x 600 mm
\end{itemize}
\subsection{Elektronik}
\begin{itemize}
\item USB-Stromversorgung
\item Mehrfachstecker
%\item Netzwerkkabel
\end{itemize}

%\thispagestyle{empty}
\end{document}
